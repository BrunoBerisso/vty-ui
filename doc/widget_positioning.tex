\section{Widget Positioning}

Some widgets, such as the \fw{Edit} widget, need to position a cursor
in the terminal when they have the focus; to support this, each widget
stores its position after it is rendered.  The positioning of the
widgets happens in a separate phase after rendering takes place, since
the positions cannot be calculated until the sizes of all widgets'
\fw{Image}s are known.

The top-level function to set a widget's position is called
\fw{setCurrentPosition} and is defined in the \fw{Core} module.  It is
called, initially, by the vty-ui event loop with a position of \fw{(0,
  0)}.  This function updates the \fw{currentPosition} field of the
widget's \fw{WidgetImpl} structure and then calls its
\fw{setCurrentPosition\_} function to take care of any widget-specific
duties.  For most widgets, \fw{setCurrentPosition\_} need not be
overridden from its default no-op implementation.  However, container
widgets must override it to set the positions of their children.

Consider the \fw{Box} widget type.  This type contains two child
widgets.  The position of the \fw{Box} itself is the upper-left corner
of the space in which it is rendered, and that position is also the
position of its first child widget.  The second child widget, however,
is offset (vertically or horizontally, depending on the box type) by
the size of the first child widget.  This is an example of a case in
which implementing \fw{setCurrentPosition\_} is necessary.

Here is an example implementation of \fw{setCurrentPosition\_} for the
\fw{Wrapper} widget that we examined in Section \ref{sec:deferring}:

\begin{minted}{haskell}
 setCurrentPosition_ = \this pos -> do
   -- Since the position of the wrapper has already been set,
   -- we just need to set the position of the child.
   (Wrapper child) <- getState this
   setCurrentPosition child pos
\end{minted}

The function calls the top-level \fw{setCurrentPosition} on the child
widget to ensure that its position is set and that its
\fw{setCurrentPosition\_} function is called.  It uses the position of
the wrapper, \fw{pos}, as the position of the child because the
wrapper has not done anything to offset that position (e.g., by adding
an ASCII art border or padding).

If you're implementing a container widget with more than one child,
you can use functions in the \fw{Util} module to manage the
\fw{DisplayRegion}s used to position your widgets; for more
information, see the \fw{withWidth}, \fw{withHeight}, \fw{plusWidth},
and \fw{plusHeight} functions.
