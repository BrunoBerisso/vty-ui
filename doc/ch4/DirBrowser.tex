\section{The Directory Browser}

The \fw{DirBrowser} module provides a rich interface for browsing the
filesystem to select files.  The user is presented with an interface
in which different file types are given different colors, and a status
bar shows some information about the currently-selected file or
directory.  If the user attempts to browse an unreadable directory or
get information about an unreadable file, an error is displayed in the
browser interface.

The \fw{DirBrowser} uses a \fw{List} widget for selecting files and
directories, so its keybindings apply here.  In total, the directory
browser supports the following key bindings:

\begin{itemize}
\item \fw{Enter} -- descends into a directory or selects a file.
\item \fw{Left} -- ascends to the parent directory.
\item \fw{Right} -- descends into a directory.
\item \fw{Up}, \fw{Down} -- changes the currently-selected entry.
\item \fw{'q'}, \fw{Esc} -- cancels browsing.
\item \fw{'r'} -- refreshes the browser's state of the current
  directory.
\end{itemize}

\fw{DirBrowser}s are created as follows:

\begin{haskellcode}
 browser <- newDirBrowser defaultBrowserSkin
\end{haskellcode}

The browser's initial filesystem path will be the application's
current directory.  You can change it with the \fw{setDirBrowserPath}
function:

\begin{haskellcode}
 setDirBrowserPath browser "/"
\end{haskellcode}

To be notified when the user has selected a file, register an event
handler with \fw{onBrowseAccept}.  The handler will be passed the
\fw{FilePath} to the file which was selected.

\begin{haskellcode}
 browser `onBrowseAccept` \path -> ...
\end{haskellcode}

Similarly, to be notified when the user has cancelled browsing,
register an event handler with \fw{onBrowseCancel}.  The handler will
be passed the browser path at the time of cancellation.

\begin{haskellcode}
 browser `onBrowseCancel` \path -> ...
\end{haskellcode}

To be notified when the user changes the browser's current path, use
\fw{onBrowserPathChange}.  The event handler will be passed the new
browser path.

\begin{haskellcode}
 browser `onBrowserPathChange` \path -> ...
\end{haskellcode}

\subsection{Skinning}

When creating a \fw{DirBrowser}, we pass it a \fw{BrowserSkin}.  This
value affects how the browser colors the different types of filesystem
entries it displays in addition to how it colors the rest of its
interface.  You can customize the browser skin by updating any of its
fields with Vty attributes of your choosing.

\begin{haskellcode}
 browser <- newDirBrowser $ defaultBrowserSkin { ... }
\end{haskellcode}

The attribute fields of the \fw{BrowserSkin} type are as follows:

\begin{itemize}
\item \fw{browserHeaderAttr} -- used for the header and footer of the
  browser interface.
\item \fw{browserUnfocusedSelAttr} -- used for the selected entry when
  the browser is not focused.
\item \fw{browserErrorAttr} -- used for the text widget which displays
  errors encountered while browsing.
\item \fw{browserDirAttr} -- used for directories.
\item \fw{browserLinkAttr} -- used for symbolic links.
\item \fw{browserBlockDevAttr} -- used for block device files.
\item \fw{browserNamedPipeAttr} -- used for named pipes.
\item \fw{browserCharDevAttr} -- used for character device files.
\item \fw{browserSockAttr} -- used for sockets.
\end{itemize}

When the browser is focused, it uses the \fw{RenderContext}'s
\fw{focusAttr} for the currently-selected entry.

\subsection{Annotations}

For each type of file on the filesystem, the browser displays the kind
of file in addition to some information about it.  For example, for
regular files, the size is displayed.  For symbolic links, the link
target is displayed.

It may be important to add your own such enhancements to the browser.
For example, you may want to apply an attribute to files with a
specific extension and make them easy to see in the browser.  In
addition, you may wish to generate a description about the file in the
status bar.  To accomplish this, the \fw{DirBrowser} provides
\textit{annotations}.

An annotation is made up of three components:

\begin{itemize}
\item A predicate to determine whether the annotation should apply to
  a given file,
\item A function to generate a description of the file such as its
  size or application-specific metadata, and
\item An attribute to apply to files of this type in the browser
  listing.
\end{itemize}

Annotations are stored in the \fw{BrowserSkin} itself since they are
used to influence the browser's appearance.  To add annotations to a
skin, use \fw{withAnnotations}.  The following example adds an
annotation for ``emacs backup files,'' which end in \fw{'\string~'}:

\begin{haskellcode}
 let mySkin = defaultBrowserSkin `withAnnotations` myAnnotations
     myAnnotations = [ ( \path _ -> "~" `isSuffixOf` path
                       , \_ _ -> return "emacs backup file"
                       , green `on` blue
                       )
                     ]
\end{haskellcode}

\subsection{Error Reporting}

When a user selects a file in the browser, your application may
determine that the file does not meet certain requirements.  At this
point it may be useful to report an error to the user without leaving
the browser interface.  The \fw{DirBrowser} provides a function to do
just this, called \fw{reportBrowserError}.  The function displays an
error message in the browser's error message area.

\begin{haskellcode}
 browser `onBrowseAccept` \path ->
   reportBrowserError browser $ "not a valid document: " ++ path
\end{haskellcode}

\subsection{Growth Policy}

A \fw{DirBrowser} expands both vertically and horizontally.
