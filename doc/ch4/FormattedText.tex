\section{Text}

The \fw{Text} module provides a widget for rendering text strings in
user interfaces.  The text widget type, \fw{Widget FormattedText}, can
be used to render simple strings or more complex text arrangements.

A \fw{FormattedText} widget can be created from a \fw{String} with the
\fw{plainText} function and can be laid out in the usual way:

\begin{haskellcode}
 t1 <- plainText "blue" >>= withNormalAttribute (fgColor blue)
 t2 <- plainText "green" >>= withNormalAttribute (fgColor green)
 ui <- (return t1) <++> (return t2)
\end{haskellcode}

\subsection{Formatters}

In addition to rendering plain text strings, we can use ``formatters''
to change the arrangement and attributes of text.  Formatters can
manipulate the the token lists and attributes to change the text
layout and appearance.

To use a formatter with a text widget, we must use a different
constructor function, \fw{textWidget}:

\begin{haskellcode}
 t <- textWidget "foobar" wrap
\end{haskellcode}

When formatters are applied, the text is automatically broken up into
``tokens,'' each of which indicates sequences of whitespace and
non-whitespace characters.  Each token stores its own attribute, and
it is these tokens on which formatters operate.

The \fw{Text} module provides two formatters: \fw{wrap} and
\fw{highlight}.  \fw{wrap} wraps the text to fit into the
\fw{DisplayRegion} available at rendering time.  \fw{highlight} uses
the \fw{pcre-light}\footnote{\fw{pcre-light} on Hackage:
  \href{http://hackage.haskell.org/package/pcre-light-0.3.1.1}{http://hackage.haskell.org/package/pcre-light-0.3.1.1}}
to highlight text using a regular ``Perl-compatible'' expression.  To
construct a highlighting formatter, we must provide the regular
expression used to match strings as well as the attribute that should
be applied to the matching strings:\footnote{Since formatters operate
  on individual tokens, the \fw{highlight} formatter applies its
  regular expression to each token individually, so it will only ever
  match sequences of characters in each token rather than matching
  more than one token.}

\begin{haskellcode}
 let doHighlight = highlight (compile (BS8.pack "bar") [])
                     (fgColor bright_green)
 t <- textWidget "Foo bar baz" doHighlight
\end{haskellcode}

Formatters can be composed with the \fw{\&.\&} operator.  This
operation results in a new formatter which will apply the operand
formatters in the specified order.  We can use this operator to
compose the built-in formatters on a single \fw{FormattedText} widget:

\begin{haskellcode}
 t <- textWidget "Foo bar baz" (doHighlight \&.\& wrap)
\end{haskellcode}

\subsection{Growth Policy}

\fw{FormattedText} widgets do not grow horizontally or vertically.
