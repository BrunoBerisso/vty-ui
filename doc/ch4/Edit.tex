\section{Edit Widgets}
\label{sec:edit}

The \fw{Edit} module provides a line-editing widget, \fw{Widget Edit}.
This widget makes it possible to edit text with some Emacs-style key
bindings.

An \fw{Edit} widget is simple to create.  You can create \fw{Edit}
widgets in two modes: single- and multi-line:

\begin{haskellcode}
 -- Single-line text editor:
 e1 <- editWidget
 -- Multi-line text editor:
 e2 <- multiLineEditWidget
\end{haskellcode}

\fw{Edit} widgets can be laid out in the usual way:

\begin{haskellcode}
 e <- editWidget
 b <- (plainText "Enter a string: ") <++> (return e)
\end{haskellcode}

To use an \fw{Edit} widget, add it to your interface and
\fw{FocusGroup}.

\fw{Edit} widgets support the following editing key bindings:

\begin{itemize}
\item \fw{Ctrl-a}, \fw{Home} -- go to the beginning of the line.
\item \fw{Ctrl-e}, \fw{End} -- go to the end of the line.
\item \fw{Ctrl-k} -- remove the text from the cursor position to the
  end of the line.
\item \fw{Ctrl-d}, \fw{Del} -- delete the character at the cursor
  position.
\item \fw{Left}, \fw{Right}, \fw{Up}, \fw{Down} -- change the cursor
  position.
\item \fw{Backspace} -- delete the character just before the cursor
  position and move the cursor position back by one character.
\item \fw{Enter} -- ``activate'' the \fw{Edit} widget if it is a
  single-line widget; if it is multi-line, insert a new line at the
  cursor position.
\end{itemize}

Note that \fw{Tab} will not be handled by \fw{Edit} widgets because it
is used to change focus.

An \fw{Edit} widget can be monitored for three events:

\begin{itemize}
\item ``Activation'' events -- triggered when the user presses
  \fw{Enter} in a single-line \fw{Edit} widget.  Handlers are
  registered with the \fw{onActivate} function.  Event handlers
  receive the \fw{Edit} widget as a parameter.
\item Text change -- when the contents of the \fw{Edit} widget change.
  Handlers are registered with the \fw{onChange} function.  Event
  handlers receive the new \fw{String} value in the \fw{Edit} widget.
\item Cursor movement -- when the cursor position within the \fw{Edit}
  widget changes.  Handlers are registered with the \fw{onCursorMove}
  function.  Event handlers receive the new cursor position as a
  parameter.
\end{itemize}

In addition to event handling, the \fw{Edit} widget API also provides
other functions.  These functions trigger the respective events
automatically.

\begin{itemize}
\item \fw{setEditText}, \fw{getEditText} -- change the current text
  content of the \fw{Edit} widget.
\item \fw{getEditCursorPosition}, \fw{setEditCursorPosition} --
  manipulate the cursor position within the \fw{Edit} widget.
\item \fw{getEditLineLimit}, \fw{setEditLineLimit} -- manipulate the
  limit on the number of lines that the text widget may hold.  Takes
  \fw{Maybe Int} where \fw{Nothing} indicates no limit.
  \fw{setEditLineLimit \$ Just 0} is a no-op.
\end{itemize}

\subsubsection{Growth Policy}

Single-line \fw{Edit} widgets -- those created by \fw{editWidget} --
grow only horizontally and are always one row high.  Multi-line edit
widgets -- those created by \fw{multiLineEditWidget} -- always grow in
both dimensions.  To manage this behavior, you can use one of the
``fixed'' family of widgets to control their sizes (see Section
\ref{sec:fixed}).
