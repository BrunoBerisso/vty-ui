\section{Borders}
\label{sec:borders}

The \fw{Borders} module provides a number border widgets which can be
created with the following functions:

\begin{itemize}
\item \fw{vBorder} -- creates a vertical border of type \fw{Widget
  VBorder}
\item \fw{hBorder} -- creates a horizontal border of type \fw{Widget
  HBorder}
\item \fw{bordered} -- creates a bordered box of type \fw{Widget
  (Bordered a)} around a widget of type \fw{Widget a}
\end{itemize}

All bordered widgets use the \fw{RenderContext}'s \fw{Skin} as
described in Section \ref{sec:skinning}.  By default, all borders will
use the \fw{RenderContext}'s normal attribute, but all border widget
types are instances of the \fw{HasBorderAttr} type class.  This type
class makes it possible to specify the border attribute of these
widgets with the \fw{setBorderAttribute} function.

The following example creates an interface using all three border
widget types.

\begin{haskellcode}
 b1 <- (plainText "foo") <--> hBorder <--> (plainText "bar")
 b2 <- (return b1) <++> vBorder <++> (plainText "baz")
 b3 <- bordered b2
\end{haskellcode}

Using the \fw{Box} combinators, we lay out text widgets separated by
different kinds of borders and wrap the entire interface in a
line-drawn box.

When drawn with the \fw{asciiSkin}, this will result in the following
interface:

\begin{verbatim}
+-------+
|foo|baz|
|---|   |
|bar|   |
+-------+
\end{verbatim}

Horizontal and box borders support labels in their top borders.  To
set the label on an \fw{HBorder}, use the \fw{setHBorderLabel}
function; for \fw{Bordered} widgets, use \fw{set\-Bordered\-Label}.
Using the example above, we can set the label on \fw{b3} to \fw{"x"}
to achieve the following result:

\begin{haskellcode}
 setBorderedLabel b3 "x"
\end{haskellcode}

\begin{verbatim}
+-- x --+
|foo|baz|
|---|   |
|bar|   |
+-------+
\end{verbatim}

If the \fw{Bordered} widget is not large enough to show the title, it
is hidden and a horizontal border is drawn instead.

\subsection{Growth Policy}

\fw{VBorder}s grow only vertically and are one column in width.
\fw{HBorder}s grow only horizontally and are one row in height.  Box
borders created with \fw{bordered} inherit the growth policies of
their children.
