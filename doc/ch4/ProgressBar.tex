\section{Progress Bars}
\label{sec:progress_bars}

The \fw{ProgressBar} module provides the \fw{ProgressBar} type which
you can use to indicate task progression in your applications.

\fw{ProgressBar}s can be created with the \fw{newProgressBar}
function.  The function takes two \fw{Color} arguments indicating the
colors to be used for the complete and incomplete portions of the
progress bar, respectively:

\begin{haskellcode}
 bar <- newProgressBar blue white
\end{haskellcode}

\fw{ProgressBar}s are composite widgest; to lay them out in your
applications, use the \fw{progressBarWidget} function:

\begin{haskellcode}
 ui <- (plainText "Progress: ") <--> (return $ progressBarWidget bar)
\end{haskellcode}

A \fw{ProgressBar} tracks progress as an \fw{Int} $0 \le n \le 100$.
To set a \fw{ProgressBar}'s progress value, use \fw{setProgress} or
\fw{addProgress}:

\begin{haskellcode}
 setProgress bar 35
 addProgress bar 1
\end{haskellcode}

Calls to \fw{setProgress} and \fw{addProgress} resulting in a progress
value outside the allowable range will have no effect.

To be notified when a \fw{ProgressBar}'s value changes, use the
\fw{onProgressChange} function.  Handlers for this event will receive
the new progress value:

\begin{haskellcode}
 bar `onProgressChange` \newVal -> ...
\end{haskellcode}

\fw{ProgressBar}s are best used with the \fw{schedule} function
described in Section \ref{sec:concurrency}.

\subsection{Growth Policy}

\fw{ProgressBar}s grow horizontally but do not grow vertically.
