\section{Composing Widgets}

As with any user interface toolkit, vty-ui lets you compose your widgets
to create a user interface that is laid out the way you want.  Widgets
fall into two basic categories:

\begin{itemize}
\item "basic" widgets like text strings, ASCII decorations
      (e.g. vertical and horizontal borders), and space-filling
      widgets
\item "container" widgets which hold other widgets and control how
      those widgets are laid out and rendered.  Most of these widgets
      influence layout; some modify other behaviors.
\end{itemize}

The most important widgets used in interface layout are the box layout
widgets:

\begin{verbatim}
vBox :: Widget a -> Widget b -> IO (Widget (Box a b))
hBox :: Widget a -> Widget b -> IO (Widget (Box a b))
\end{verbatim}

The vBox function returns a new widget of type 'Widget (Box a b)' which
lays out its two children vertically in the order they are passed to the
function.  The hBox function does the same for horizontal layout.  These
two widgets will probably be the most common in your applications.

vty-ui provides some combinators to make Boxes a bit eaiser to work
with:

\begin{verbatim}
(<-->) :: IO (Widget a) -> IO (Widget b) -> IO (Widget (Box a b))
(<++>) :: IO (Widget a) -> IO (Widget b) -> IO (Widget (Box a b))
\end{verbatim}

These functions are essentially aliases for vBox and hBox, respectively,
with the important difference being that they take IO arguments; so, for
example, you can use them to create nested boxes as follows:

\begin{verbatim}
mainBox <- (hBox a b) <--> (hBox c d <++> vBox e f)
\end{verbatim}

If you already have a pure reference to another widget, you can merely
wrap it with 'return' to use it with these combinators:

\begin{verbatim}
box1 <- hBox a b
box2 <- (return box1) <++> (hBox c d)
\end{verbatim}

The box layout widgets do more than merely place their children next to
each other.  Box widgets determine how to lay their children out
depending on two primary factors:

\begin{itemize}
\item the amount of terminal space available to the box at the time it
      is rendered
\item the size policies of the child widgets
\end{itemize}

Just as with graphical toolkits, when the terminal is resized, more
space is available to render the interface, so we need to use the space
wisely, and to determine how to use it, vty-ui requires that the widgets
declare their own policies for how to use the available space.  The box
size policy is to expand to use all available space to provide the
maximum space to its children.  For more details on how Box widget is
implemented, see the API documentation.

Just placing text widgets in boxes won't get you very far; to influence
layout, you'll need to add other space-filling widgets to those boxes to
dynamically take advantage of the space in the terminal.  See the API
documentation for the 'Fills' module in particular.

There are many other examples of widgets which influence their children;
take note of widget constructors which take other widgets as parameters.
