\section{Text Clipping}
\label{sec:textclip}

Most widgets in \vtyui\ render some form of text.  When we render
text, we have to reason about the amount of space -- columns -- that
the text will consume in the terminal so that we can render coherent
interfaces.  However, this is tricky because some characters use one
column of space and others use
two.\footnote{\url{http://www.unicode.org/reports/tr11/}} To account
for this possibility, any code which deals with computing the space
required for text must consider the width of \textit{each character}.

In cases where we have to consider character width, the most common
operation we're trying to perform is to \textit{clip} a text string to
ensure that it fits within a given region.  The \fw{TextClip} module
provides types and functions to do this in one and two dimensions:

\begin{itemize}
\item \fw{ClipRect} - the type of two-dimensional clipping regions.
  Allows you to specify a top-left corner, a clipping width, and a
  clipping height.
\item \fw{clip1d} - performs one-dimensional clipping on a single line
  of text.
\item \fw{clip2d} - performs two-dimensional clipping on a list of
  lines of text using a \fw{Clip\-Rect}.
\end{itemize}

The functions in the \fw{TextClip} module deal in physical values
expressed using the \fw{Phys} type.  This type designates a
\textit{physical width} as opposed to a logical one.  We use this
distinction to gain compile-time clarity about which integer values
refer to logical characters and which ones refer to terminal column
counts.

Both the \fw{clip1d} and \fw{clip2d} functions return text strings
truncated so that their characters fit into the specified physical
space.  They also return \fw{Bool} indicators which can be used to
determine whether clipping occurred in the ``middle'' of wide
characters.  The \fw{Edit} widget uses this feature to annotate
truncated strings to indicate that a wide character can be found on
either end of a truncated line of text.

In simple widgets, we could technically ignore text clipping details
if we know that we'll always be rendering strings which use
single-column characters.  However, we should get in the habit of
always using clipping functions in case we need to start showing
multi-column characters.
