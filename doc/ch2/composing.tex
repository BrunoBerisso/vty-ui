\section{Composing Widgets}

As with any user interface toolkit, \vtyui\ lets you compose your
widgets to create a user interface that is laid out the way you want.
Widgets fall into two basic categories:

\begin{itemize}
\item ``Basic'' widgets, such as text strings, ASCII decorations
  (e.g. vertical and horizontal borders), and space-filling widgets.
\item ``Container'' widgets, which hold other widgets and control how
  those widgets are laid out and rendered.  Most of these widgets
  influence layout; some modify other behaviors.
\end{itemize}

The most important widgets used in interface layout are the box layout
widgets:

\begin{haskellcode}
 vBox :: Widget a -> Widget b -> IO (Widget (Box a b))
 hBox :: Widget a -> Widget b -> IO (Widget (Box a b))
\end{haskellcode}

The \fw{vBox} returns a \fw{Box} widget which lays out its two
children vertically in the order in which they are passed to the
function.  The \fw{hBox} function does the same for horizontal layout.
These two widget types will probably be the most common in your
applications.

\vtyui\ provides some combinators to make \fw{Box}es a bit eaiser to
work with:

\begin{haskellcode}
 (<-->) :: IO (Widget a) -> IO (Widget b) -> IO (Widget (Box a b))
 (<++>) :: IO (Widget a) -> IO (Widget b) -> IO (Widget (Box a b))
\end{haskellcode}

These functions are essentially aliases for \fw{vBox} and \fw{hBox},
respectively, with the important difference being that they take
\fw{IO} arguments.  You can use them to create nested boxes as
follows:

\begin{haskellcode}
 mainBox <- (hBox a b) <--> (hBox c d <++> vBox e f)
\end{haskellcode}

If you already have a reference to another widget, you can merely wrap
it with \fw{return} to use it with these combinators:

\begin{haskellcode}
 box2 <- (return box1) <++> (hBox c d)
\end{haskellcode}

The box layout widgets do more than merely place their children next
to each other.  \fw{Box} widgets determine how to lay their children
out depending on two primary factors:

\begin{itemize}
\item the amount of terminal space available to the box at the time it
      is rendered
\item the size policies of the child widgets
\end{itemize}

Just as with graphical toolkits, when the terminal is resized, more
space is available to render the interface, so we need to use the
space wisely.  To determine how to use it, \vtyui\ requires that the
widgets declare their own policies for how to use the available space.
The size policy for the \fw{Box} itself is to expand to use all
available space only if that is true for either of its children.  As a
result, a \fw{Box} containing two fixed-size widgets will have a fixed
size.  For more details on how the \fw{Box} widget is implemented, see
the API documentation.

Placing text widgets in \fw{Box}es may suffice for most purposes.  See
the documentation for space-filling widgets for greater control over
box layout.

There are many other examples of widgets which influence their
children; we'll see more examples of these in Chapter
\ref{chap:guided_tour}.
