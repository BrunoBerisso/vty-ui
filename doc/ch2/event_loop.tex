\section{The \vtyui\ Event Loop}
\label{sec:event_loop}

\vtyui\ manages the user input event loop for you, and once you have
created and populated a \fw{Collection}, you can invoke the main
\vtyui\ event loop:

\begin{haskellcode}
 runUi c defaultContext
\end{haskellcode}

The first parameter is the \fw{Collection} you have created; the
second parameter is a \fw{Ren\-der\-Con\-text}.  Here we use the
``default'' rendering context provided by the library.  The
``rendering context'' provides three key pieces of functionality:

\begin{itemize}
\item The "skin" to use when rendering ASCII lines, corners, and
      intersections
\item The default ``normal'' (unfocused) attribute
\item The default ``focused'' attribute
\item The current ``override'' attribute
\end{itemize}

The event loop will run until one of two conditions occurs:

\begin{itemize}
\item An exception of any kind is thrown; if an exception is thrown,
  the event loop will shut down Vty cleanly and re-throw the
  exception.
\item An event handler or thread calls \fw{shutdownUi}; the
  \fw{shutdownUi} function sends a signal to stop the event loop, at
  which point control will be returned to your program.  The shutdown
  signal goes into a queue with all of the other signals processed by
  the event loop, such as key input events and scheduled actions (see
  Section \ref{sec:concurrency}), but it will preempt them.  Note that
  there is no \textit{guarantee} that there won't be some other signal
  placed into the queue before you run \fw{shutdownUi}, such as when
  another thread is running in parallel with an event handler which
  calls \fw{shutdownUi}.
\end{itemize}

\subsection{Skinning}
\label{sec:skinning}

Some widgets, such as the \fw{Table} widget (see Section
\ref{sec:tables}) and the horizontal and vertical border widgets
\fw{VBorder} and \fw{HBorder} (see Section \ref{sec:borders}), use
line-drawing characters to draw borders between interface elements.
Some terminal emulators are capable of drawing Unicode characters,
which make for nicer-looking line-drawing.  Other terminal emulators
work best only with ASCII.  The default rendering context uses a
Unicode line-drawing skin, which you can change to any other skin (or
your own) as follows:

\begin{haskellcode}
 runUi c $ defaultContext { skin = asciiSkin }
\end{haskellcode}

The library provides \fw{Skin}s in the \fw{Skins} module.

\subsection{Attributes}
\label{sec:attributes}

An attribute may consist of one or more settings of foreground and
background color and text style, such as underline or blink.  The
default attributes specified in the \fw{Render\-Context} control how
widgets appear.

Every widget has the ability to store its own normal and focused
attributes.  When widgets are rendered, they use these attributes; if
they are not set, the widgets default to using those specified by the
rendering context.  The only exception is the ``override'' attribute.
Instead of ``falling back'' to this attribute, the presence of this
attribute requires widgets to use it.  For example, this attribute is
used in the \fw{List} widget so that the currently-selected list item
can be highlighted, which requires the \fw{List} to override the
item's default attribute configuration.

Widgets provide an API for setting these attributes using the
\fw{Has\-Normal\-Attr} and \fw{Has\-Focus\-Attr} type classes.  The
reason we use type classes to provide this API is so that third-party
widgets may also provide this functionality.  The API is defined in
the \fw{Core} module and is as follows:

\begin{haskellcode}
 setNormalAttribute w attr
 setFocusAttribute w attr
\end{haskellcode}

Convenience combinators also exist:

\begin{haskellcode}
 w <- someWidget
      >>= withNormalAttribute attr
      >>= withFocusAttribute attr
\end{haskellcode}

The \fw{attr} value is a Vty attribute.  A Vty attribute may provide
any (but not necessarily all!) of the settings that make up an
attribute; any setting not specified (e.g. background color) can fall
back to the default.  As a result, the attribute of a widget is the
\textit{combination} of its attribute and the attribute from the
rendering context.  The widget's settings will take precedence, but
any setting not provided will default to the rendering context.

Consider this example:

\begin{haskellcode}
 w <- someWidget
 setNormalAttribute w (fgColor white)
 runUi c $ defaultContext { normalAttr = yellow `on` blue }
\end{haskellcode}

In this example, the widget \fw{w} will use a normal attribute of
white on a blue background, since it specified only a foreground color
as its normal attribute.  This kind of precedence facilitates visual
consistency across your entire interface.

In addition, container widgets are designed to pass their normal and
focused attributes onto their children during the rendering process;
this way, unless a child specifies a default with
\fw{setNormalAttribute} or similar, it uses its parent's attributes.
Again, this facilitates consistency across the interface while only
requiring the you to specify attributes where you want to deviate from
the default.

You can create attributes with varying levels of specificity by using
the \vtyui\ API:

\begin{tabular}{|l|l|} \hline
Expression & Resulting attribute \\ \hline
\fw{fgColor blue} & foreground only \\ \hline
\fw{bgColor blue} & background only \\ \hline
\fw{style underline} & style only \\ \hline
\fw{blue `on` red} & foreground and background \\ \hline
\fw{someAttr `withStyle` underline} & adding a style \\ \hline
\end{tabular}

The Vty \fw{def\_attr} value's default configuration is used as a
basis for all partially-specified attributes.  The functions described
above are defined in the \fw{Util} module.

\subsection{\vtyui\ and Concurrency}
\label{sec:concurrency}

So far we have only seen programs which modify widget state when user
input events occur.  Such changes in widget state are safe, because
they are triggered by the \vtyui\ event loop.\footnote{``Unsafe''
  updates are those that are not guaranteed to be reflected in the
  most-recently-rendered interface.}  However, your program will more
than likely need to trigger some widget state changes due to other
external events -- such as network events -- and \vtyui\ provides a
mechanism for doing this in a safe way.

\vtyui\ provides a function in the \fw{Core} module called
\fw{schedule} which takes an \fw{IO} action and ``schedules'' it to be
run by the main event loop.  It will be run as soon as possible, i.e.,
once the program control flow has returned to the event loop.  Since
the scheduled action will be run by the event loop, it's important
that the action not take very long; if it's important to block (e.g.,
by calling \fw{Control.Concurrent.threadDelay}), you should do that in
a thread and only call \fw{schedule} when you have work to do.

Consider this example, in which a text widget called \fw{timeText}
gets updated with the current time every second:

\begin{haskellcode}
 forkIO $
   forever $ do
     schedule $ do
       t <- getCurrentTime
       setText timeText $
         formatTime defaultTimeLocale rfc822DateFormat t
     threadDelay 1000000
\end{haskellcode}

In this example the blocking occurs outside of the scheduled code, and
only when we have an update for the clock display do we schedule an
action to run.

Some built-in widgets will almost always be used in this way; for an
example, take a look at the \fw{ProgressBar} widget in the
\fw{ProgressBar} module (see Section \ref{sec:progress_bars}).
