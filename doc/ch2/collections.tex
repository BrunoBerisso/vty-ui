\section{Collections}
\label{sec:collections}

Traditional user interfaces present the user with a window for each
task the user needs to accomplish.  Since we don't have the option of
presenting multiple "windows" to users of a terminal interface, we
must present the user with one interface at a time.  Then, through the
use of event handlers, the application will manage the transition
between these interfaces.

Consider a text editor program in which we must present these top-level
interfaces in the following order:

\begin{itemize}
\item The user runs the program and is presented with an interface to
  select a file to edit;
\item The user chooses a file to edit and is presented with the
  editing interface;
\item After editing, the user chooses to exit and we present a dialog
  which asks the user whether to save the file.
\end{itemize}

All three of these interfaces are separate and should be given the
entire terminal window; unlike other graphical toolkits, \vtyui\ does
not provide a way to "show" or "hide" widgets.  Instead, it provides
the notion of a "collection."  A \fw{Collection} is a widget which
wraps a set of other widgets and maintains a pointer to the one that
should be displayed at any given time.  The application then changes
the current interface by changing the \fw{Collection}'s state.

But an interface is more than what is presented in the terminal; each
interface should have its own set of user input widgets and its own
notion of focus.  Therefore, a \fw{Collection} is a set of interfaces
\textit{and their focus groups}.  When we change the state of the
\fw{Collection}, we are really changing both the visual interface as
well as the focus group used to interact with it.

To create a \fw{Collection}:

\begin{haskellcode}
 c <- newCollection
\end{haskellcode}

To add an interface and a \fw{FocusGroup} to the \fw{Collection}:

\begin{haskellcode}
 fg <- newFocusGroup
 -- Add widgets to focus group fg
 ui <- someWidget
 changeToW <- addToCollection c ui fg
\end{haskellcode}

As a convenience, \fw{addToCollection} returns a \fw{MonadIO} action
which, when run, will switch to the specified interface.  In the
example above, \fw{changeToW} is an action which will switch to the
interface with \fw{ui} as its top-level widget and \fw{fg} as its
focus group.  You can use this action in event handlers that change
your interface state.  If you prefer, you can use the
\fw{setCurrentEntry} function instead, which allows you to set the
\fw{Collection}'s interface by number.  Use of \fw{setCurrentEntry} is
not recommended, however, since a bad index can cause an exception to
be thrown.
