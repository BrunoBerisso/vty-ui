\section{Cursor Positioning}

Once a widget is properly positioned, the widget can display a cursor.
This is especially useful for edit widgets, since the user needs to
know the cursor position.  The \fw{Core} module provides a top-level
function to accomplish this called \fw{getCursorPosition}; this
function calls the \fw{WidgetImpl} type's \fw{getCursorPosition_}
function to query widgets.

The \fw{getCursorPosition_} function returns \fw{Maybe DisplayRegion};
a return value of \fw{Nothing} indicates that the widget does not want
to show a cursor, so when it gains focus, no cursor will be displayed.
Otherwise, positioning the cursor at row \fw{r} and column \fw{c} is
accomplished by returning \fw{Just (DisplayRegion r c)}.  The cursor
is then shown at that location by the event loop.

Typically, the position of the cursor is computed as an offset to the
widget's current position.  In Section \ref{sec:deferring} we deferred
to the child widget to control the cursor, but we might instead
specify our own position:

\begin{minted}{haskell}
getCursorPosition_ = \this -> do
  childCursor <- getCursorPosition child
  case childCursor of
    Nothing -> return Nothing
    Just pos -> return $ Just $ pos `plusWidth` 1 `plusHeight` 1
\end{minted}

Although contrived, this example shows how we can return a new cursor
position based on the child widget's cursor position.
