\section{Handling User Input}

Many widgets in \vtyui\ can accept user input.  A widget can accept
user input if (1) it has one or more key event handlers attached to it
and (2) if it currently has the ``focus''.  The concept of focus in
\vtyui\ works the same as in other user interface toolkits:
essentially, only one widget has the focus and any user input is
passed to that widget for handling.  Key event handlers can be chained
so that if one handler does not handle an input event, it can be
passed on to remaining handlers until either a handler accepts it or
all handlers decline to handle it.

Key event handlers can be added to any widget as follows:

\begin{minted}{haskell}
 w <- someWidget
 w `onKeyPressed` \this key modifiers -> do
   ...
   return False
\end{minted}

The handler must return IO Bool; True indicates that the handler
processed the key event and took action, and False indicates that the
handler declined to handle the event.  The event handler is passed the
keystoke itself along with any modifier keys detected by the underlying
Vty input processing.

Key event handlers are invoked in the order in which they are added to
the widget.  In the following example, the first handler will deline the
\fw{'q'} key but the second one will process it:

\begin{minted}{haskell}
 w `onKeyPressed` \_ key _ ->
   if key == KASCII 'f' then
     (launchTheMissiles >> return True) else
     return False

 w `onKeyPressed` \_ key _ ->
   if key == KASCII 'q' then
     exitSuccess else return False
\end{minted}

This functionality allows any widget to have its own "default" input
event handling while still allowing you to add custom input event
handling.

Although any widget -- even a basic text widget -- can accept input
events in this way, the events will only reach the widget if it has the
focus.  The way we manage focus is with "focus groups."
