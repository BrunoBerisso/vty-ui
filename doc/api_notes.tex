\section{Notes on the \texttt{vty-ui} API}

When you create a widget in vty-ui, the result with almost always have a
type like 'Widget a'.  The type variable 'a' represents the specific
type of state the widget can carry, and therefore which operations can
be performed on it.  For example, a text widget has type 'Widget
FormattedText'.

The 'Widget' type is actually an 'IORef' which wraps the real widget
implementation type, 'WidgetImpl'.  So it's safe (and recommended!) to
always use 'Widget a' whenever you need to refer to a widget; this
makes it possible to mutate widget state when events occur in your
application.

All widget constructors must ultimately be run in the 'IO' monad, so
all API functions must be run in an instance of 'MonadIO'.  In this
manual we will use 'IO' to simplify type signatures, but keep in mind
that the actual type is likely to be '(MonadIO m) => m'.  Although
'MonadIO' is by far the more common constraint, be sure to check the
API documentation to be sure; some functions, such as callbacks, are
'IO' actions.
