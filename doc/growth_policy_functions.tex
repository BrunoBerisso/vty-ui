\section{Growth Policy Functions}
\label{sec:growth_policy_functions}

In order to lay widgets out in way that makes the best use of the
available terminal space, we need them to give us hints about how they
use space.  In this regard, widgets fall into two basic categories:

\begin{itemize}
\item ``Fixed-size'' widgets which have the same size regardless of
  the amount of available space
\item ``Variable-size'' widgets which use all available space
\end{itemize}

An example of a ``fixed-size'' widget is a text widget: the string
``\fw{foobar}'' will always require only one row and six columns'
worth of space.  We could also render such a widget in a much bigger
space -- an entire terminal window, say -- but it would look the same;
there would still be plenty of room for other things in the inteface.
Such a widget does not ``grow'' with the available space.

An example of a ``variable-size'' widget is one which centers its
child widget vertically and horizontally in the terminal.  Such a
widget will pad its child widget accordingly so that it is always
centered, and this behavior depends on how much space is available.
For example, in a 100x100 terminal, the string ``\fw{foobar}'' would
need different padding to remain centered than it would require in a
50x50 terminal.  As a result, we say that the wrapper widget -- the
one that does the centering -- ``grows'' with available space.

The \fw{WidgetImpl} type defines the following functions to provide
these hints, where \fw{a} is the type of widget state:

\begin{itemize}
\item \fw{growHorizontal\_ ::\ a -> IO Bool}
\item \fw{growVertical\_ ::\ a -> IO Bool}
\end{itemize}

These functions should return \fw{True} when the widget in question
``grows'' as described above, and \fw{False} otherwise.  These hints
may be used by parent widgets to make layout decisions; concrete
examples of such widgets are the \fw{Box} and \fw{Centered} widget
types.

In situations where your widget wraps another -- as with the \fw{Box}
and \fw{Centered} types -- it is \textit{strongly} recommended that
you defer to the child widgets for these policy values \textit{unless}
you have a good reason to override them.  The \fw{Centered} widget is
a good example of this: it overrides the growth policy of its child so
that it grows in both dimensions, even though its child may not.  But
the \fw{Box} widget explicitly defers to its children to determine its
growth policy, since it is only responsible for layout and does not
add anything to the interface.

An example of a \fw{growHorizontal\_} implementation which defers to a
child widget is as follows:

\begin{haskellcode}
 growHorizontal_ = \st -> do
   let ch = getChildWidget st
   growHorizontal ch
\end{haskellcode}

Notice that we call the top-level function, \fw{growHorizontal}, on
the child widget; it does the job of dereferencing the widget and
calling its \fw{growHorizontal\_} function.  This is another example
of the API convention we mentioned in Section
\ref{note:api_convention_1}. (XXX fix this ref, since it does not
actually go to the note.)
