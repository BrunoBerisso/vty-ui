\section{Composite Widgets}

So far we have looked at single-purpose widgets which use the
\fw{Widget} type directly.  However, embedding widget state in the
\fw{Widget} type is not always appropriate or straightforward for more
complex, composite widgets.

The \vtyui\ library provides some ``widgets'' which don't fit this
pattern: \fw{Dialog} and \fw{DirBrowser} are two examples.
Furthermore, as the base set of widgets provided by the library
becomes richer, fewer and fewer widgets should be implemented using
the basic \fw{Widget} framework.

These composite widgets are actually entire interfaces, complete with
multiple focusable widgets and focus groups.  These widgets don't take
the form of \fw{Widget Dialog} or \fw{Widget DirBrowser}; they
\textit{could} be implemented that way, but we'd find that many of the
\fw{WidgetImpl} functions would end up deferring to their child
widgets anyway, and their \fw{render\_} implementations would be
cumbersome at best.

Instead, we invert the widget organization: we create a type (e.g.,
\fw{Dialog}) which contains the actual widget(s) to be rendered, as
well as other book-keeping internals, and we return that from our
constructor.  This makes it easier to implement such widgets since we
are less concerned with their inner workings and more concerned with
returning something high-level that has the right behaviors.

The pattern we use in these situations is to write a constructor which
does all of the widget creation, layout, and event handler
registration, and returns the concrete type of the interface along
with a \fw{FocusGroup} which the caller can use to integrate the
interface into an application.

For example: suppose we want to create a ``phone number input'' widget
-- \fw{PhoneInput}, say -- which will allow users to input phone
numbers.  The \fw{PhoneInput} will have three \fw{Edit} widgets and
will manage tabbing between them and might even do such things as data
validation on the input.  Here's a suggestive example for how we might
implement such a thing without going to all the trouble of
implementing \fw{WidgetImpl}'s interface.  First we provide the types:

\begin{haskellcode}
 data PhoneNumber = PhoneNumber String String String
                    deriving (Show)

 -- This type isn't pretty, but we have to specify the type
 -- of the complete interface.  Initially you can let the
 -- compiler tell you what it is.
 type T = Box (Box
               (Box (Box (HFixed Edit) FormattedText) (HFixed Edit))
               FormattedText) (HFixed Edit)

 data PhoneInput =
   PhoneInput { phoneInputWidget :: Widget T
              , edit1 :: Widget Edit
              , edit2 :: Widget Edit
              , edit3 :: Widget Edit
              , activateHandlers :: Handlers PhoneNumber
              }
\end{haskellcode}

Then, we provide the constructor:

% Let this block span a page boundary since it's so big that it's
% likely, and we don't want to bump it down unless it looks good. :)
\begin{haskellcode*}{samepage=false}
 newPhoneInput :: IO (PhoneInput, Widget FocusGroup)
 newPhoneInput = do
   ahs <- newHandlers
   e1 <- editWidget
   e2 <- editWidget
   e3 <- editWidget

   ui <- (hFixed 4 e1) <++>
         (plainText "-") <++>
         (hFixed 4 e2) <++>
         (plainText "-") <++>
         (hFixed 5 e3)

   setEditMaxLength e1 3
   setEditMaxLength e2 3
   setEditMaxLength e3 4

   e1 `onChange` \s -> when (length s == 3) $ focus e2
   e2 `onChange` \s -> when (length s == 3) $ focus e3

   let w = PhoneInput ui e1 e2 e3 ahs
       doFireEvent = const $ do
         num <- mkPhoneNumber
         fireEvent w (return . activateHandlers) num

       mkPhoneNumber = do
         s1 <- getEditText e1
         s2 <- getEditText e2
         s3 <- getEditText e3
         return $ PhoneNumber s1 s2 s3

   e1 `onActivate` doFireEvent
   e2 `onActivate` doFireEvent
   e3 `onActivate` doFireEvent

   fg <- newFocusGroup
   mapM_ (addToFocusGroup fg) [e1, e2, e3]
   return (w, fg)
\end{haskellcode*}

Then we provide a function to register phone number handlers:

\begin{haskellcode}
 onPhoneInputActivate :: PhoneInput
                      -> (PhoneNumber -> IO ()) -> IO ()
 onPhoneInputActivate input handler =
   addHandler (return . activateHandlers) input handler
\end{haskellcode}

When the user presses \fw{Enter} in one of the phone number input
widgets, thus ``activating'' it, we will invoke all phone number input
handlers with a \fw{PhoneNumber} value.\footnote{Assume that we would
  also do some kind of validation and decide whether to call the
  handlers accordingly.  We might even consider supporting ``error''
  event handlers for the widget to report validation errors to be
  displayed elsewhere in the interface!}

In the calling environment, the caller can then add the
\fw{phoneInputWidget} to the interface and merge the returned
\fw{FocusGroup} as described in Section
\ref{sec:merging_focus_groups}.
