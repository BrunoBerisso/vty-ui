\section{Notes on the \vtyui\ API}

When you create a widget in \vtyui, the result with almost always have
a type like \fw{Widget a}.  The type variable \fw{a} represents the
specific type of state the widget can carry, and therefore which
operations can be performed on it.  For example, a text widget has
type \fw{Widget FormattedText}.

The \fw{Widget} type is actually an \fw{IORef} which wraps the real
widget implementation type, \fw{WidgetImpl a}.  So it's best to use
\fw{Widget a} whenever you need to refer to a widget; this makes it
possible to mutate widget state when events occur in your application.

All widget constructors must ultimately be run in the \fw{IO} monad,
so all API functions must be run in an instance of \fw{MonadIO}.  In
this manual we will use \fw{IO} to simplify type signatures, but keep
in mind that the actual type is likely to be \fw{(MonadIO m) => m}.
Although \fw{MonadIO} is by far the more common constraint, be sure to
check the API documentation to be sure; some functions, such as event
handlers, are \fw{IO} actions.

Throughout this document, we'll refer frequently to widgets by their
state type (e.g., ``\fw{Edit} widgets''). In most cases we are
referring to a value whose type is, e.g., \fw{Widget Edit}.  When in
doubt, be sure to check the API documentation.
