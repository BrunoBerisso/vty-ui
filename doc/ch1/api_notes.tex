\section{Conventions and API Notes}

When you create a widget in \vtyui, the result with almost always have
a type like \fw{Widget a}.  The type variable \fw{a} represents the
specific type of state the widget can carry, and therefore which
operations can be performed on it.  For example, a text widget has
type \fw{Widget FormattedText}.  Throughout this document, we'll refer
frequently to widgets by their state type (e.g., ``\fw{Edit}
widgets''). In most cases we are referring to a value whose type is,
e.g., \fw{Widget Edit}.  When in doubt, be sure to check the API
documentation.

The \fw{Widget} type is actually an \fw{IORef} which wraps the real
widget implementation type, \fw{WidgetImpl a}.  So it's best to use
\fw{Widget a} whenever you need to refer to a widget; this makes it
possible to mutate widget state when events occur in your application.

Regarding return values, even if a function is of type \fw{...\ -> IO
  a}, we say it is ``in the \fw{IO} monad'' and \textit{returns}
\fw{a}.  We won't bother saying that a function \textit{returns \fw{IO
    a}}.

Lastly, we will refer to the many \vtyui\ library modules throughout
this document.  We will almost always omit the
\fw{Graphics.Vty.Widgets} module namespace prefix and will instead
refer to the modules by their short names.
