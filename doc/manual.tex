\documentclass[11pt, letterpaper, oneside, titlepage]{article}
\renewcommand{\rmdefault}{ppl}
\renewcommand{\ttdefault}{pcr}
\usepackage{hyperref}
\hypersetup{colorlinks,citecolor=blue,%
            filecolor=red,linkcolor=blue,%
            urlcolor=blue}
\addtolength{\hoffset}{-0.5in}
\addtolength{\voffset}{-0.5in}

\setlength{\topmargin}{0in}

% Adjust the width of the body; increase it by 1.5in.
\addtolength{\textwidth}{1.0in}
\setlength{\parskip}{0.1in}
\setlength{\parindent}{0in}

\begin{document}
\vsize 9.0in

\title{\texttt{vty-ui} User's Manual}
\author{
  For \texttt{vty-ui} version 0.5\\
  Jonathan Daugherty\\
  \href{mailto:jtd@galois.com}{jtd@galois.com}
}
\maketitle

\tableofcontents
\newpage

\section{Introduction}

The terminal emulator user interface is a good, lightweight
alternative to fully graphical interfaces such as those provided by
GTK, QT, and the Windows and Macintosh OS X operating systems.  Aside
from being lightweight, such interfaces are appealing because they can
be used for remote administration more easily, and many users prefer
them over graphical interfaces for their responsiveness.

Historically, terminal interfaces have been notoriously difficult to
program.  Libraries have appeared to aid in this task, such as
Ncurses.  Ncurses is fairly low-level so other, higher-level libraries
arose, such as CDK, Dialog, and Newt.  This library fits into the same
category as those high-level libraries.

vty-ui provides a “widget” infrastructure for constructing user
interfaces similar to what you will find in libraries such as QT and
GTK.  In addition to rendering infrastructure, vty-ui provides
infrastructure for managing user input events, changes in widget focus,
box layout support, and a flexible API for binding event handlers to
widget events.  It is built on the Vty library, which provides
functionality similar to Ncurses.

This manual provides comprehensive documentation on how to use
vty-ui's built-in widgets as well as information on how to implement
your own.

\section{Getting Started}

To get started using the library, you'll need to import the main library
module:

\begin{verbatim}
import Graphics.Vty.All
\end{verbatim}

This module imports everything exported by the library; if you prefer,
you may import specific modules depending on your needs.

As a demonstration, we'll create a program which presents an editing
widget in the middle of the screen.  You'll be able to provide some
text input and press Enter, at which point the program will exit and
will print what you entered.  The code for this program is as follows:

\begin{verbatim}
main :: IO ()
main = do
  e <- editWidget
  ui <- centered e

  fg <- newFocusGroup
  addToFocusGroup fg e

  c <- newCollection
  addToCollection c ui fg

  e `onActivate` \this ->
    getEditText this >>= (error . ("You entered: " ++))

  runUi c defaultContext
\end{verbatim}

There are some interesting things to note about this program.  First,
it withstands changes in your terminal size automatically, even though
the size of the terminal is not an explicit part of the program.
Second, it only took a few lines of code to create a rich editing
interface and position it in the terminal as desired.  Now we'll go
into some depth on this example.

\begin{verbatim}
e <- editWidget
\end{verbatim}

This line creates an “edit widget.”  This type of widget provides an
editing interface for a single line of text and supports some
Emacs-style editing keybindings.  For more information on this widget
type, see XXX.

\begin{verbatim}
ui <- centered e
\end{verbatim}

This creates a new widget, ui, which centers the edit widget vertically
and horizontally.  This is a common pattern: create one widget and wrap
it in another to affect its behavior.  For more information on
centering, see XXX.

\begin{verbatim}
fg <- newFocusGroup
\end{verbatim}

This creates a “focus group.”  A focus group is an ordered sequence of
widgets that will receive focus as you cycle between them.  By default,
this is done with the Tab key.  Every vty-ui interface requires a focus
group.

\begin{verbatim}
addToFocusGroup fg e
\end{verbatim}

This adds the edit widget to the focus group.  The first widget to be
added to a focus group automatically receives the initial focus, and
widgets receive focus in the order they are added to the group.

\begin{verbatim}
c <- newCollection
\end{verbatim}

This creates a new “collection.”  A collection is group of interfaces,
each with its own focus group, and the collection makes it possible to
switch between these interfaces.  Think of an e-mail client whose
initial interface might be listing the contents of the inbox; subsequent
interactions might change the interface to present only the selected
message on the screen, with different navigation keystrokes, one of
which returns to the inbox interface.  Collections make it easy to
switch between such interface modes.  Every vty-ui program requires a
collection.

\begin{verbatim}
addToCollection ui fg
\end{verbatim}

This adds the top-level user interface widget, ui, to the collection and
sets its focus group to fg.  This means that the widgets to receive the
user’s focus (and, consequently, input) will be those in the focus group
fg and the interface to be presented will be ui.

\begin{verbatim}
e `onActivate` \this -> getEditText this >>=
  (error . ("You entered: " ++))
\end{verbatim}

This binds an event handler to the “activation” of the edit widget,
which occurs when the user focuses the edit widget and presses Enter.
The handler for this event is an IO action which takes the edit widget
itself as its only parameter.  The getEditText function gets the current
text of the edit widget, and we use error to abort the program and print
the text.

\begin{verbatim}
runUi c defaultContext
\end{verbatim}

This runs the main vty-ui event loop with the collection we created
above.  We pass a “default rendering context” which provides defaults
for the rendering process, such as the default foreground and background
colors to be used for normal and focused widgets, as well as a “skin”
for line-drawing.  The main event loop processes input events from the
Vty library and re-draws the interface after calling any event handlers.
It also shuts down Vty in the event of an exception.

We've now seen the general structure of a vty-ui program:
\begin{itemize}
\item Create and compose widgets,
\item Create a focus group and add input-receiving widgets to the group,
\item Create a collection and add the top-level widget and focus group
      to the collection, and
\item Invoke the main event loop with the collection and some default
      rendering settings.
\end{itemize}

\section{The vty-ui API}

When you create a widget in vty-ui, the result with almost always have a
type like 'Widget a'.  The type variable 'a' represents the specific
type of state the widget can carry, and therefore which operations can
be performed on it.  For example, a text widget has type 'Widget
FormattedText'.

All widget constructors must ultimately be run in the IO monad, so all
API functions must be run in an instance of MonadIO.  In this manual, we
will use IO to simplify type signatures.  Although MonadIO is by far the
more common constraint, be sure to check the API documentation to be
sure; some functions, such as callbacks, are IO actions.

\section{Composing Widgets}

As with any user interface toolkit, vty-ui lets you compose your widgets
to create a user interface that is laid out the way you want.  Widgets
fall into two basic categories:

\begin{itemize}
\item "basic" widgets like text strings, ASCII decorations
      (e.g. vertical and horizontal borders), and space-filling
      widgets
\item "container" widgets which hold other widgets and control how
      those widgets are laid out and rendered.  Most of these widgets
      influence layout; some modify other behaviors.
\end{itemize}

The most important widgets used in interface layout are the box layout
widgets:

\begin{verbatim}
vBox :: Widget a -> Widget b -> IO (Widget (Box a b))
hBox :: Widget a -> Widget b -> IO (Widget (Box a b))
\end{verbatim}

The vBox function returns a new widget of type 'Widget (Box a b)' which
lays out its two children vertically in the order they are passed to the
function.  The hBox function does the same for horizontal layout.  These
two widgets will probably be the most common in your applications.

vty-ui provides some combinators to make Boxes a bit eaiser to work
with:

\begin{verbatim}
(<-->) :: IO (Widget a) -> IO (Widget b) -> IO (Widget (Box a b))
(<++>) :: IO (Widget a) -> IO (Widget b) -> IO (Widget (Box a b))
\end{verbatim}

These functions are essentially aliases for vBox and hBox, respectively,
with the important difference being that they take IO arguments; so, for
example, you can use them to create nested boxes as follows:

\begin{verbatim}
mainBox <- (hBox a b) <--> (hBox c d <++> vBox e f)
\end{verbatim}

If you already have a pure reference to another widget, you can merely
wrap it with 'return' to use it with these combinators:

\begin{verbatim}
box1 <- hBox a b
box2 <- (return box1) <++> (hBox c d)
\end{verbatim}

The box layout widgets do more than merely place their children next to
each other.  Box widgets determine how to lay their children out
depending on two primary factors:

\begin{itemize}
\item the amount of terminal space available to the box at the time it
      is rendered
\item the size policies of the child widgets
\end{itemize}

Just as with graphical toolkits, when the terminal is resized, more
space is available to render the interface, so we need to use the space
wisely, and to determine how to use it, vty-ui requires that the widgets
declare their own policies for how to use the available space.  The box
size policy is to expand to use all available space to provide the
maximum space to its children.  For more details on how Box widget is
implemented, see the API documentation.

Just placing text widgets in boxes won't get you very far; to influence
layout, you'll need to add other space-filling widgets to those boxes to
dynamically take advantage of the space in the terminal.  See the API
documentation for the 'Fills' module in particular.

There are many other examples of widgets which influence their children;
take note of widget constructors which take other widgets as parameters.

\section{Handling User Input}

Many widgets in vty-ui can accept user input.  A widget can accept user
input if (1) it has one or more key event handlers attached to it and
(2) if it currently has the 'focus'.  The concept of focus in vty-ui
works the same as in other user interface toolkits:  fundamentally, only
one widget has the focus, and any user input is passed to that widget
for handling.  Key event handlers can be chained so that if one handler
does not handle an input event, it can be passed on to remaining
handlers until either a handler accepts it or all handlers decline to
handle it.

Key event handlers can be added to any widget as follows:

\begin{verbatim}
w <- someWidget
w `onKeyPressed` {\backslash}this key modifiers -> do
  ...
\end{verbatim}

The handler must return IO Bool; True indicates that the handler
processed the key event and took action, and False indicates that the
handler declined to handle the event.  The event handler is passed the
keystoke itself along with any modifier keys detected by the underlying
Vty input processing.

Key event handlers are invoked in the order in which they are added to
the widget.  In the following example, the first handler will deline the
'q' key but the second one will process it:

\begin{verbatim}
w `onKeyPressed` \backslash _ key _ ->
  if key == KASCII 'f' then
    (launchTheMissiles >> return True) else
    return False

w `onKeyPressed` \backslash _ key _ ->
  if key == KASCII 'q' then
    exitSuccess else return False
\end{verbatim}

This functionality allows any widget to have its own "default" input
event handling while still allowing you to add custom input event
handling.

Although any widget -- even a basic text widget -- can accept input
events in this way, the events will only reach the widget if it has the
focus.  The way we manage focus is with "focus groups."

\section{Focus Groups and Focus Changes}

Graphical interfaces allow the user to change focus between all of the
primary interface input elements, usually with the Tab key.  The same is
true in vty-ui, except that because any widget can accept events -- and
because which widgets the focus is a policy decision that you get to
make -- the library cannot automatically determine which widgets should
get the focus, or the order in which focus should be received.  As a
result, vty-ui provides a type called a "focus group."

A focus group is just an ordered sequence of widgets that should get the
user's focus as the Tab key is pressed.  Widgets receive focus in the
order in which they are added to the group, and the first widget to be
added automatically gets the focus when the interface is set up.

Creating a focus group is simple:

\begin{verbatim}
fg <- newFocusGroup
\end{verbatim}

Adding widgets to focus groups is also straightforward:

\begin{verbatim}
w <- someWidget
addToFocusGroup fg w
\end{verbatim}

A widget's "focused behavior" depends entirely on the widget's
implementation.  Some widgets, when focused, provide a text cursor;
others merely change foreground and background color.  In any case, the
widgets that the user can interact with should be in the interface's
focus group.

Once the widgets are added to the focus group, you won't have to manage
anything else; the Tab key event is intercepted by the FocusGroup
itself, and user input events are passed to the focused widget until the
focus is changed.

If, for some reason, you would like to be notified when a widget
receives or loses focus, you may register event handlers for these
events on any widget:

\begin{verbatim}
w <- someWidget
w `onGainFocus` {\backslash}this -> ...
w `onLoseFocus` {\backslash}this -> ...
\end{verbatim}

In both cases above, the 'this' parameter to each event handler is just
the widget to which the event handler is being attached (i.e., 'w').
Many event handlers follow this pattern.

\section{Top-Level Key Event Handlers}

All user input is handled via a focus group; the focus state of the
group indicates which widget will receive user input events.  However,
focus groups are widgets, too!  Although they cannot be rendered, they
support the same key handler interface as other widgets.  This is how we
create "top-level" key event handlers for the interface.  For example,
if you want to register a handler for a "quit" key such as 'q', the
focus group itself is where this key event handler belongs.  This is
because focus groups always try to handle key events first, and only
pass those events onto the focused widget if they have no registered
handler.

\begin{verbatim}
fg <- newFocusGroup
fg `onKeyPressed` \backslash _ key _ ->
  if key == KASCII 'q' then
    exitSuccess else return False
\end{verbatim}

\section{Merging Focus Groups}

Some widgets, such as the "dialog" widget, are composed of a number of
input widgets already; these widgets must create their own focus groups
and will return them to you when they are created.  In order to
integrate these focus groups' input widgets into your overall
application, you must merge them with your own focus group.

For example, consider the "directory browser" widget provided by the
library; you might want to place this alongside other widgets that
should also accept input.  When you create the browser widget, you will
get a reference to the widget and a reference to its focus group.

\begin{verbatim}
(browser, fg1) <- newDirBrowser defaultBrowserSkin

fg2 <- newFocusGroup
-- Add my own widgets to fg2

merged <- mergeFocusGroups fg1 fg2
\end{verbatim}

The 'mergeFocusGroups' function will merge the two focus groups and
preserve the order of the widgets, such that widgets in the first group
will come before widgets in the second group in the new group's focus
ordering.  The merged group should then be passed to the rest of the
setup process that we introduced in Getting Started; we'll go into more
detail on that in the next section.

\section{Collections}

Traditional user interfaces present the user with a window for each task
the user needs to accomplish.  Since we don't have the option of
presenting multiple "windows" to users of a terminal interface, we must
present the user with one interface at a time.  Then, the application
will manage the transition between these interfaces.

Consider a text editor program in which we must present these top-level
interfaces in the following order:

\begin{itemize}
\item The user runs the program and is presented with an interface to
      select a file to edit
\item The user chooses a file to edit and is presented with the
      editing interface
\item After editing, the user chooses to exit and we present a dialog
      which asks the user whether to save the file
\end{itemize}

All three of these interfaces are separate and should be given the
entire terminal window; unlike other graphical toolkits, vty-ui does not
provide a way to "show" or "hide" widgets.  Instead, it provides the
notion of a "collection."  A collection is a widget which wraps a set of
other widgets and maintains a pointer to the one that should be
displayed at any given time.  The application then changes the current
interface by changing the collection state.

But an interface is more than what is presented in the terminal; each
interface should have its own set of user input widgets and its own
notion of focus.  Therefore, a collection is a set of interfaces *and
their focus groups.*  When we change the state of the collection, we are
really changing both the visual interface as well as the focus group
used to interact with it.

To create a collection:

\begin{verbatim}
c <- newCollection
\end{verbatim}

To add an interface and a focus group to the collection:

\begin{verbatim}
fg <- newFocusGroup
-- Add widgets to focus group fg
ui <- someWidget
changeToW <- addToCollection c ui fg
\end{verbatim}

As a convenience, addToCollection returns a MonadIO action which, when
run, will switch to the specified interface.  In the example above,
changeToW is an action which will switch to the interface with 'ui' as
its top-level widget and 'fg' as its focus group.  You can use this
action in event handlers that change your interface state.  If you
prefer, you can use the 'setCurrentEntry' function instead, which allows
you to set the collection's interface by number.  That method is not
recommended, however, since a bad index can cause an exception to be
thrown.

\section{The vty-ui Event Loop}

Vty-ui manages the user input event loop for you, and once you have
created and populated a collection, you can invoke the main vty-ui event
loop:

\begin{verbatim}
runUi c defaultContext
\end{verbatim}

The first parameter is the collection you have created; the second
parameter is the 'rendering context'.  Here we use the 'default'
rendering context provided by the library.  The rendering context
provides three key pieces of functionality:

\begin{itemize}
\item The "skin" to use when rendering ASCII lines, corners, and
      intersections
\item The default "normal" (unfocused) attribute
\item The default "focused" attribute
\end{itemize}

\subsection{Skinning}

Some widgets, such as the table widget and the horizontal and vertical
border widgets, use line-drawing characters to draw borders between
interface elements.  Some terminal emulators are capable of drawing
Unicode characters, which make for nicer-looking line-drawing.  Other
terminal emulators work best only with ASCII.  The default rendering
context uses a Unicode line-drawing skin, which you can change to any
other skin (or your own) as follows:

\begin{verbatim}
runUi c $ defaultContext { skin = asciiSkin }
\end{verbatim}

The library provides skins in the 'Skin' module.

\subsection{Attributes}

The default attributes specified in the skin control how widgets appear.
An attribute may consist of one or more settings of foreground and
background color and text style, such as underline or blink.

Every widget has the ability to store its own normal and focused
attributes.  When widgets are rendered, they use these attributes; if
they are not set, the widgets default to using those specified by the
rendering context.  Widgets provide an API for setting these attributes
using the 'HasNormalAttr' and 'HasFocusAttr' type classes.  The reason
we use type classes to provide this API is so that third-party widgets
may also provide this functionality.  The API is defined in the 'Core'
module and is as follows:

\begin{verbatim}
setNormalAttribute w attr
setFocusAttribute w attr
\end{verbatim}

Convenience combinators also exist:

\begin{verbatim}
w <- someWidget
     >>= withNormalAttribute attr
     >>= withFocusAttribute attr
\end{verbatim}

The 'attr' value is a Vty attribute value.  A Vty attribute may provide
any (but not necessarily all!) of the settings that make up an
attribute; any setting not specified (e.g. background color) can fall
back to the default.  As a result, the attribute of a widget is the
*combination* of its attribute and the attribute from the rendering
context.  The widget's settings will take precedence, but any setting
not provided will default to the rendering context.

Consider this example:

\begin{verbatim}
w <- someWidget
setNormalAttribute w (fgColor white)

-- ... add the widget to a collection, etc.
runUi c $ defaultContext { normalAttr = yellow `on` blue }
\end{verbatim}

In this example, the widget 'w' will use a normal attribute of white on
a blue background, since it specified only a foreground color as its
normal attribute.  This kind of precedence allows you to very easily
maintain consistency across your entire interface.

In addition, container widgets are designed to pass their normal and
focused attributes onto their children during the rendering process;
this way, unless a child specifies an override with 'setNormalAttribute'
or similar, it uses its parent's attributes.  Again, this facilitates
consistency across the interface while only requiring the you to specify
attributes where you want to deviate from the default.

You can create attributes with varying levels of specificity by using
the vty-ui API:

\begin{verbatim}
fgColor blue -- foreground only
bgColor blue -- background only
style underline -- style only
blue `on` red -- foreground and background
someAttr `withStyle` underline -- adding a style
\end{verbatim}

The Vty 'def\_attr' value's default configuration is used as a basis for
all partially-specified attributes.  The functions described above are
defined in the 'Util' module.
\end{document}
